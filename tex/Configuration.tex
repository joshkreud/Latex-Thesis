\documentclass[11pt,a4paper,numbers=enddot,bibliography=totoc,toc=sectionentrywithdots,dvipsnames]{scrartcl}
\pdfminorversion=7
\usepackage[utf8]{inputenc} % Formatting
\usepackage{subfiles} % Subfiles
\usepackage[T1]{fontenc} %Europäische fontcodierung
\usepackage[ngerman]{babel} % Deutsch
\usepackage{textcomp} % Euro Zeichen usw für T1 font
\usepackage{lmodern} %Clear PDF Fonts
\usepackage[onehalfspacing]{setspace} % Zeilenabstand
\usepackage[font=scriptsize,labelfont=bf]{caption} % Caption modifications
\usepackage{amsmath}
\usepackage{lipsum}


%\setlength{\leftmargini}{0pt} %Description nicht einrücken

% \usepackage[titles]{tocloft}
%\addtolength{\cftsecnumwidth}{10pt}

%------------------------ Schnellkommandos (Formatierung) ----------------------
\newcommand*{\name}[1]{\textsc{#1}}%
\newcommand*{\verweis}[1]{\textcolor{Black70}{\texttt{#1}}}%
\newcommand*{\fachbegriff}[1]{\emph{#1}}%
\newcommand*{\wichtig}[1]{\textbf{#1}}%
\newcommand*{\zB}{\mbox{z.\,B.\ }}%
\newcommand*{\email}[1]{\textcolor{MidnightBlue}{\href{mailto:#1}{\ttfamily#1}}}%

% Registered definieren
\def\SymbReg{\textsuperscript{\textregistered}}

% Unterschriftenboxen
\newcommand{\titledate}[2][2.5in]{%
	\noindent%
	\begin{tabular}{@{}p{#1}@{}}
		\\\\ \hline \\[-.75\normalbaselineskip]
		#2
	\end{tabular} \hspace{1in}
	\begin{tabular}{@{}p{#1}@{}}
		\\\\ \hline \\[-.75\normalbaselineskip]
		Ort / Datum
	\end{tabular}
}


\newcommand{\Milestone}[2]{
	%\par
	\begin{tabularx}{\textwidth}{m{2cm}X}
		%	\\ \relax
		\includegraphics[width=2cm]{Meilenstein.pdf} & \parbox[c]{\hsize}{Meilenstein \enquote{\wichtig{\name{#1}}} erreicht: #2}
	\end{tabularx}
}

%------------------------ SI Units and better Number Display ----------------------
\usepackage{siunitx} % mathematitsche einheiten. / TausenderTrennzeichen
%SI Unit Setup
\sisetup{
	locale=DE,
	seperr,
	repeatunits=true,
	separate-uncertainty,
	per-mode =fraction,
	trapambigerr=false,
	tophrase={{ bis }},
	list-units = brackets,
	%range-units = brackets,
	multi-part-units = brackets,
	table-unit-alignment = left,
	group-separator={.},
	detect-all=true
	%detect-family=true
}


%------------------------ Todos &Draft ----------------------
%\usepackage[german,figwidth=10cm,shadow]{todonotes}
%\usepackage{draftwatermark}
%\definecolor{lightRed}{RGB}{250,189,189}
%\SetWatermarkColor{lightRed}

%------------------------ Schriftarten ----------------------
%\usepackage{mathpazo} %Schöne font

% Oder Arial:
\usepackage{helvet}
\renewcommand{\familydefault}{\sfdefault}

%------------------------ Bilder Einfügen ----------------------

\usepackage{graphicx} % Bilder einfügen
\usepackage{subcaption}
\captionsetup[subfigure]{subrefformat=simple,labelformat=simple}
\captionsetup{font=normalsize}
\captionsetup[sub]{font=small}
\renewcommand\thesubfigure{(\alph{subfigure})}

\usepackage{wrapfig} % Fließende Bilder im Tex
\usepackage{float}

\usepackage{chngcntr} % Fußnote fortlaufend
\makeatletter
\@addtoreset{figure}{section}
\@addtoreset{table}{section}
\makeatother

\renewcommand{\thefigure}{\thesection.\arabic{figure}} %Bilder mit Section Nummer Prefixed
\renewcommand{\thetable}{\thesection.\arabic{table}}   %Tabellen mit Section Nummer Prefixed



%Image Pfad
\graphicspath{{./img/}{../img/}}

%------------------------ Better Textformatting ----------------------
\usepackage[activate={true,nocompatibility},babel,protrusion=true,expansion=true,final,tracking=true,kerning=true,spacing=true,stretch=10,shrink=10]{microtype}
\SetProtrusion{encoding={*},family={bch},series={*},size={6,7}}
{1={ ,750},2={ ,500},3={ ,500},4={ ,500},5={ ,500},
	6={ ,500},7={ ,600},8={ ,500},9={ ,500},0={ ,500}}
\SetExtraKerning[unit=space]
{encoding={*}, family={bch}, series={*}, size={footnotesize,small,normalsize}}
{\textendash={400,400}, % en-dash, add more space around it
"28={ ,150}, % left bracket, add space from right
"29={150, }, % right bracket, add space from left
\textquotedblleft={ ,150}, % left quotation mark, space from right
\textquotedblright={150, }} % right quotation mark, space from left

\SetExtraKerning[unit=space]
{encoding={*}, family={qhv}, series={b}, size={large,Large}}
{1={-200,-200}, \textendash={400,400}}

\SetTracking{encoding={*}, shape=sc}{40}

% Schusterjungen und Hurenkinder vermeiden
\clubpenalty = 10000
\widowpenalty = 10000
\displaywidowpenalty = 10000

%------------------------ Compiler Settings ----------------------
\begingroup\expandafter\expandafter\expandafter\endgroup
\expandafter\ifx\csname pdfsuppresswarningpagegroup\endcsname\relax
\else
	\pdfsuppresswarningpagegroup=1\relax
\fi



%------------------------ Literatur ----------------------
\usepackage[
	backend=biber,
	style=numeric, %
	isbn=false,
	sorting=none,                % ISBN nicht anzeigen, gleiches geht mit nahezu allen anderen Feldern
	pagetracker=true,          % ebd. bei wiederholten Angaben (false=ausgeschaltet, page=Seite, spread=Doppelseite, true=automatisch)
	maxbibnames=50,            % maximale Namen, die im Literaturverzeichnis angezeigt werden (ich wollte alle)
	maxcitenames=3,            % maximale Namen, die im Text angezeigt werden, ab 4 wird u.a. nach den ersten Autor angezeigt
	autocite=inline,           % regelt Aussehen für \autocite (inline=\parancite)
	block=space,               % kleiner horizontaler Platz zwischen den Feldern
	backref=true,              % Seiten anzeigen, auf denen die Referenz vorkommt
	backrefstyle=three+,       % fasst Seiten zusammen, z.B. S. 2f, 6ff, 7-10
	date=short,                % Datumsformat
]{biblatex}

% prints author names as small caps
\renewcommand{\mkbibnamefirst}[1]{\textsc{#1}}
\renewcommand{\mkbibnamelast}[1]{\textsc{#1}}
\renewcommand{\mkbibnameprefix}[1]{\textsc{#1}}
\renewcommand{\mkbibnameaffix}[1]{\textsc{#1}}


%------------------------ Zitate ----------------------
\usepackage[autostyle=true,german=quotes]{csquotes}% Better Quotes
%Autoenquote von Zitaten
\renewcommand*{\mkblockquote}[4]{\enquote{#1}#2#4#3}
\renewcommand*{\mktextquote}[4]{\enquote{#1}#2#4#3}

\bibliography{./sources}%Literaturdatei

%------------------------ Anhänge ----------------------
\usepackage{appendix}
\addto\captionsngerman{\let\appendixtocname\appendixname%
	\let\appendixpagename\appendixname}
\usepackage{pdfpages} % Pdfs EInbinden

%Insert Multisite pdfs into appendix with Section
\newcommand*{\includemultipdf}[2]{
	\includepdf[pages=1,frame= true,width=\textwidth,pagecommand={#2}, offset=0 -0.5cm]{#1}
	\includepdf[pages=2-,frame= true,width=\textwidth,pagecommand={}]{#1}
}


%------------------------ TABLES ----------------------
\usepackage{booktabs,ltablex} % Better Tables
%\usepackage{longtable,tabularx }
\textheight = 4in                      %to force a page break after a few rows
%\renewcommand{\arraystretch}{1.5}   %improve spacing
\newcolumntype{Y}{>{\centering\arraybackslash}X}

%-------------------Fix Longtable Numbering--------------------
\makeatletter
\newif\ifLT@nocaption
\preto\longtable{\LT@nocaptiontrue}
\appto\endlongtable{%
	\ifLT@nocaption
		\addtocounter{table}{\m@ne}%
	\fi}
\preto\LT@caption{%
	\noalign{\global\LT@nocaptionfalse}}
\makeatother
%--------------------------------------------------------------

\captionsetup[table]{skip=5pt} % Table Caption näher an Tabelle
\setlength\heavyrulewidth{0.25ex}

\usepackage{etoolbox} % Tabelle Numberings
\preto\tabular{\setcounter{magicrownumbers}{0}}
\newcounter{magicrownumbers}
\newcommand\rownumber{\stepcounter{magicrownumbers}\arabic{magicrownumbers}.\space}

%------------------------ Erste Section der Seite im header anzeigen ----------------------
\renewcommand{\sectionmark}[1]{
	\markright{\thesection{} #1}{} % Erste Section als Rightmark setzen
}




%------------------------ PDF Optionen ----------------------
\usepackage[
	%hidelinks,
	pdfborder={0 0 0},
	bookmarks,
	linktoc=all,
	bookmarksopen=true,
	colorlinks=false,
	linkcolor=blue, % einfache interne Verkn¸pfungen
	anchorcolor=black, % Ankertext
	citecolor=red, % Verweise auf Literaturverzeichniseintr‰ge im Text
	filecolor=black, % Verkn¸pfungen, die lokale Dateien ˆffnen
	menucolor=black, % Acrobat-Men¸punkte
	urlcolor=blue,
	pagebackref=false,
	plainpages=false, % zur korrekten Erstellung der Bookmarks
	pdfpagelabels, % zur korrekten Erstellung der Bookmarks
	%hypertexnames=false, % zur korrekten Erstellung der Bookmarks
]{hyperref}
%\usepackage[all]{hypcap}
\usepackage{caption}

% Befehle, die Umlaute ausgeben, f¸hren zu Fehlern, wenn sie hyperref als Optionen ¸bergeben werden
\hypersetup{
	pdftitle={\titel - \untertitel},
	pdfauthor={\autor},
	pdfcreator={\autor},
	pdfsubject={\untertitel},
	pdfkeywords={\titel}{\untertitel},
}

%------------------------ Farben ----------------------
\usepackage{xcolor} % Own color
\usepackage{colortbl}
\definecolor{Red}{RGB}{177,0,52}
\definecolor{Black70}{RGB}{112,113,115}
\definecolor{Black40}{RGB}{177,179,180}
\definecolor{Black20}{RGB}{217,218,219}
\definecolor{Black10}{RGB}{236,237,237}

%------------------------ Formatierung ----------------------
\addtokomafont{section}{\color{Red}}
\addtokomafont{subsection}{\color{Red}}
\addtokomafont{subsubsection}{\color{Red}}
\renewcommand\UrlFont{\color{MidnightBlue}\ttfamily}
% Paragraph start nicht einrücken
\setlength{\parindent}{0pt}


%------------------------ Referencing Must be Loaded after hyperref ----------------------
\usepackage[ngerman,nameinlink]{cleveref}

\newcommand*{\cnameref}[1]{(s. \cref{#1} \nameref{#1})}
%\newcommand*{\Cnameref}[1]{(s. \Cref{#1} \nameref{#1})}

\newcommand{\mycref}[2][]{(s. \cref{#2}{#1})}
%\newcommand{\myCref}[2][]{(s. \Cref{#2}{#1})}


\makeatletter
\AtBeginDocument{%
	\if@cref@abbrev\crefname{table}{Tab.}{Tab.}%
	\else\Crefname{table}{Tabelle}{Tabellen}\fi%
	\if@cref@abbrev\crefname{appendix}{Anh.}{Anh.}%
	\else\Crefname{appendix}{Anhang}{Anhänge}\fi%
}%
\makeatother

%Abbreviations for Captions (Geht nur in KOMA Script klassen)
\renewcaptionname{ngerman}{\contentsname}{Inhalt}
\renewcaptionname{ngerman}{\listfigurename}{Abbildungen}
\renewcaptionname{ngerman}{\listtablename}{Tabellen}
\renewcaptionname{ngerman}{\figurename}{Abb.}
\renewcaptionname{ngerman}{\tablename}{Tab.}

%------------------------ Pagelayout ----------------------
\usepackage[left=30mm,right=30mm,top=30mm,bottom=30mm,footskip=.8cm]{geometry} % Seitenränder & Layout

%------------------------ Kopf-Fußzeile ----------------------
\usepackage[headsepline,footsepline,plainfootsepline,automark,pagestyleset=KOMA-Script,]{scrlayer-scrpage} %Haeder & Footer
\usepackage{lastpage} %Pagenumbers
\usepackage{setspace}
\automark{section}
\automark*{subsection}
\clearpairofpagestyles

\pagestyle{scrheadings}
\setlength{\headheight}{45pt}
\renewcommand{\headfont}{\normalfont}
\rohead{\includegraphics[width=2.5cm]{Testbild.jpg}}
\lohead{\textbf{\titel} \\ \parbox[b]{\textwidth /2}{\begin{spacing}{0,8} \color{Red}\headmark \end{spacing}} }
\cohead{}
\lofoot{\scriptsize\autor\ }
\cofoot{\today}


%------------------------ Glossaries ----------------------
\usepackage[acronyms,translate=babel,xindy,style=list]{glossaries}

\renewcommand*{\glossarypreamble}{%
	\label{\currentglossary}%
}

\renewcommand*{\glossaryentrynumbers}[1]{Seite: #1}
%\defglsentryfmt[main]{\fachbegriff{\Glsentryname{\glslabel}}\footnote{\Glsentrydesc{\glslabel}}}

\usepackage{xparse}
\DeclareDocumentCommand{\newdualentry}{ O{} O{} m m m m } {
	\newglossaryentry{gls-#3}{name={#5 (#4)},description={#6},#1}
\makeglossaries
\newglossaryentry{#3}{type=\acronymtype, name={#4}, description=#5, first={\gls{gls-#3}}, see=[Glossar:]{gls-#3}}
}

%\renewcommand{\glsnumberformat}[1]{%
%	(\ifthenelse{\DTLisSubString{#1}{\delimN}\OR\DTLisSubString{#1}{\delimR}}%
%	{Seiten #1}{Seite #1})}


\loadglsentries[main]{./tex/Glossaries}
\makeglossaries